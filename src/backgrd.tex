\chapter*{Background}
The portable energy sources, such as rechargeable batteries, supercapacitors, and fuel cells, are critical to economic development and human's daily life. Among the several types of rechargeable batteries, lithium-ion batteries (LIBs) are the most promising for their highest performance: high energy density; long cycle life; low rate of self-discharge; environmental pollution. Besides, LIBs are also more flexible in terms of design for a wide range of applications in portable electronic devices, being able to be manufactured in numerous sizes and shapes to make effective use of spaces. Since the commercialization of LIBs in 1991, LIBs have been utilized in laptops, cellar phones, and even larger-scale electric vehicles (EVs). 

Especially, to tackle the dilemma of rapid fossil fuel depletion and environmental pollution, governments all around the world have introduced policies to implement the transition from a fossil-based economy to a zero-carbon economy in the next decade, making the EVs crucial to realize such transition. Therefore, the EVs sales worldwide have soared in recent five years. In 2010, only about 17,000 electric cars were on the world’s roads. But by 2019, that number has swelled to 7.2 million, 47\% of which are in China. The future EVs market is of great potential.
However, the further commercialization of EVs is still hindered by several technological barriers principally from LIBs, including high costs, insufficient cycle lives, intrinsically poor safety characteristics, and poor performances at low temperatures.  

Based on the above background, this study focus on investigating next generation LIBs from a material's point of view.
