\chapter{Conclusion}
The present study was carried out aiming to fabricate novel Silicon-based anode  for next-generation LIBs.  To attain high energy density, Silicon should be nano-sized to prevent its rapid volume expansion, carbonous materials are often used as additives to buffer the volume change as well as enhance the conductivity. Among all carbonous materials, carbon nanotubes is the material of interest for it’s large aspect ratio and high electrical conductivity. The method to fabricate nano-sized Silicon is PS-PVD for the high throughput and the feasibility of industrial-scale production. To make full use of the Si:Ni nanostructure where Ni directly attached on Si nanoparticle, a low-temperature CNT growth method was utilized. 

In this study,
\begin{enumerate}[(1)]
\item A low-temperature CNT growth mathematical model was established.
\item The competitive process between Ni silicidation and carbon deposition was simulated. By modeling the CNT growth and Ni silicidation, the proper experimental condition was set.
\item However, the experimental result reveals that the product of annealing is a uniform carbonous coating instead of CNT. Still, the mathematical model is ready to simulate the carbon coating thickness.
\item The correlation between silicide, amorphous carbon coating and battery performance was evaluated. There is a trade-off: the higher temperature, the more carbonous product on Si:Ni-NP surface to increase capacity. But the Ni silicidation is also promoted, which consumes Si, then decrease the capacity. And Ni silicide decrease conductivity of battery as well.
\end{enumerate}

So far, the CNT growth mechanism remains unclear,  and introducing plasma add up the complexity to quantify CNT growth. PE-CVD of CNTs is an extremely complex process with numerous coupled phenomena: plasma chemistry, neutral and ion reactions, surface chemistry, catalyzed growth, catalyst particle aggregation and poison, additional plasma heating, electric field effects, ion bombardment. The model established in this study use only major two parameters: Temperature and carbon concentration. Although the initial purpose is to grow CNT at low temperature, this model and simulation is applicable for other forms of carbon. 
